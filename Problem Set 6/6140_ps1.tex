% Class Notes Template
\documentclass[12pt]{article}
\usepackage[margin=1in]{geometry} 
\usepackage[utf8]{inputenc}

% Packages
\usepackage[french, english]{babel}
\usepackage{amsmath,amsthm,amssymb,amsfonts, amsopn,mathrsfs, epstopdf, graphicx,color, tikz, float, enumerate}
\usepackage{listings}
\usepackage{color} %red, green, blue, yellow, cyan, magenta, black, white
\definecolor{mygreen}{RGB}{28,172,0} % color values Red, Green, Blue
\definecolor{mylilas}{RGB}{170,55,241}

\lstset{language=Matlab,%
	%basicstyle=\color{red},
	breaklines=true,%
	morekeywords={matlab2tikz},
	keywordstyle=\color{blue},%
	morekeywords=[2]{1}, keywordstyle=[2]{\color{black}},
	identifierstyle=\color{black},%
	stringstyle=\color{mylilas},
	commentstyle=\color{mygreen},%
	showstringspaces=false,%without this there will be a symbol in the places where there is a space
	numbers=left,%
	numberstyle={\tiny \color{black}},% size of the numbers
	numbersep=9pt, % this defines how far the numbers are from the text
	emph=[1]{for,end,break},emphstyle=[1]\color{blue}, %some words to emphasise
	%emph=[2]{word1,word2}, emphstyle=[2]{style},    
}

% Title
\title{ECON 6140 - Problem Set \# 1}
\date{\today}
\author{Julien Manuel Neves}

% Use these for theorems, lemmas, proofs, etc.
\newtheorem{theorem}{Theorem}
\newtheorem{corollary}[theorem]{Corollary}
\newtheorem{lemma}[theorem]{Lemma}
\newtheorem{observation}[theorem]{Observation}
\newtheorem{proposition}[theorem]{Proposition}
\newtheorem{definition}[theorem]{Definition}
\newtheorem{claim}[theorem]{Claim}
\newtheorem{fact}[theorem]{Fact}
\newtheorem{assumption}[theorem]{Assumption}
\newtheorem{problem}[theorem]{Problem}
\newtheorem{set-up}[theorem]{Set-up}
\newtheorem{example}[theorem]{Example}
\newtheorem{remark}[theorem]{Remark}
\newtheorem{axiom}[theorem]{Axiom}

% Usefuls Macros
\newcommand{\field}[1]{\mathbb{#1}}
\newcommand{\N}{\field{N}} % natural numbers
\newcommand{\R}{\field{R}} % real numbers
\newcommand{\Z}{\field{Z}} % integers
\newcommand\F{\mathcal{F}}
\newcommand\B{\mathbb{B}}
\renewcommand{\Re}{\R} % reals
\newcommand{\Rn}[1]{\mathbb{R}^{#1}}
\newcommand{\1}{{\bf 1}} % vector of all 1's
\newcommand{\I}[1]{\mathbb{I}_{\left\{#1\right\}}} % indicator function
\newcommand{\La}{\mathscr{L}}
% \newcommand{\tends}{{\rightarrow}} % arrow for limits
% \newcommand{\ra}{{\rightarrow}} % abbreviation for right arrow
% \newcommand{\subjectto}{\mbox{\rm subject to}} % subject to

%% math operators
\DeclareMathOperator*{\argmin}{arg\,min}
\DeclareMathOperator*{\argmax}{arg\,max}
\DeclareMathOperator*{\maximize}{maximize}
\DeclareMathOperator*{\minimize}{minimize}
\DeclareMathOperator{\E}{\mathsf{E}} % expectation
\newcommand{\Ex}[1]{\E\left\{#1\right\}} % expectation with brackets
\DeclareMathOperator{\pr}{\mathsf{P}} % probability
\newcommand{\prob}[1]{\pr\left\{#1\right\}}
\DeclareMathOperator{\subjectto}{{s.t.\ }} % subject to
\newcommand{\norm}[1]{\left\|#1\right\|}
\newcommand{\card}[1]{\left|#1\right|}

% Extra stuff
\newcommand\seq[1]{\{ #1 \}}
\newcommand{\inp}[2]{\langle #1, #2 \rangle}

\newcommand{\inv}{^{-1}}

\newcommand{\pa}[1]{\left(#1\right)}
\newcommand{\bra}[1]{\left[#1\right]}
\newcommand{\cbra}[1]{\left\{ #1 \right\}}

\newcommand{\pfrac}[2]{\pa{\frac{#1}{#2}}}
\newcommand{\bfrac}[2]{\bra{\frac{#1}{#2}}}

\newcommand{\mat}[1]{\begin{matrix}#1\end{matrix}}
\newcommand{\pmat}[1]{\pa{\mat{#1}}}
\newcommand{\bmat}[1]{\bra{\mat{#1}}}

\begin{document}

\maketitle

\section*{Problem 1}
Let $r_t^c$ and $r_t^i$ be the rental rate of capital for the consumption good and the investment good sectors respectively. Moreover, let $w_t^c$ and $w_t^i$ be the wage for the consumption good and the investment good sectors respectively.

Hence, the household problem is the following
		\begin{align*}
	\max_{c_t,k_{t+1}^i,k_{t+1}^c,n_{t}^i,n_{t}^c} \sum_{t=0}^{\infty}\beta^t u(c_t,1-n_t) \\
	\subjectto p_t^cc_t+p_t^ii_t &\leq w_t^cn_t^c+w_t^in_t^i+ r_t^ck_t^c+r_t^ik_t^i  \\
i_t &=	k_{t+1}^c+k_{t+1}^i-(1-\delta)(k_t^c+k_t^i)  \\
n_t &=	n_t^c+n_t^i  \\
c_t,k_{t+1}^i,k_{t+1}^c,n_{t}^i,n_{t}^c	& \geq 0
	\end{align*}
	
We can obtain the following combined FOCs:
	\begin{align*}	
	c_t: & u_c(c_t,1-n_t) - \lambda_t  p_t^c =0 \\
	k_{t+1}^i:& -\lambda_tp_t^i +\beta \lambda_{t+1}(r_{t+1}^i+p_{t+1}^i(1-\delta)) =0\\
	k_{t+1}^c:&-\lambda_tp_t^i + \beta \lambda_{t+1}(r_{t+1}^c+p_{t+1}^i(1-\delta)) =0\\
	n_t^i: & -u_l(c_t,1-n_t) + \lambda_t  w_t^i =0 \\
	n_t^c: & -u_l(c_t,1-n_t) + \lambda_t  w_t^c =0 \\
	&p_t^cc_t+p_t^i(k_{t+1}^c+k_{t+1}^i-(1-\delta)(k_t^c+k_t^i)) =w_t^cn_t^c+w_t^in_t^i+ r_t^ck_t^c+r_t^ik_t^i \\
 & n_t =	n_t^c+n_t^i
	\end{align*}
	
	First, note that the FOCs implies that $w_t^i=w_t^c$. Moreover, we have that $-\lambda_tp_t^i +\beta \lambda_{t+1}(r_{t+1}^i+p_{t+1}^i(1-\delta)) =-\lambda_tp_t^i + \beta \lambda_{t+1}(r_{t+1}^c+p_{t+1}^i(1-\delta))\Rightarrow r_t^i=r_t^c$.
	
	For simplicity, let $w_t^i=w_t^c=w_t$ and $r_t^i=r_t^c=r_t$.
	
	Since the firms are CRS, the marginal product of the inputs are equal to their marginal cost in both sectors. Hence, we have that
	\begin{align*}
		p_t^cz_t^cf_k(k_t^c,n_t^c)&=r=p_t^iz_t^if_k(k_t^i,n_t^i)\\
		p_t^cz_t^cf_n(k_t^c,n_t^c)&=w=p_t^iz_t^if_n(k_t^i,n_t^i)
	\end{align*}
	
	Thus, by the previous set of equations and CRS, we get
	\begin{align*}
		\frac{f_n\left( \frac{k_t^c}{n_t^c},1\right) }{f_k\left( \frac{k_t^c}{n_t^c},1\right)} = \frac{w}{r} = \frac{f_n\left( \frac{k_t^i}{n_t^i},1\right)}{f_k\left( \frac{k_t^i}{n_t^i},1\right)} \\
	\end{align*}

Since the strict concavity of $f$ implies that $f_n\left( x,1\right)$ is strictly increasing in $x$ and $f_n\left( x,1\right)$ is strictly decreasing in $x$. Thus, $\frac{f_n\left( x,1\right) }{f_k\left( x,1\right)}$ is strictly increasing in $x$ and therefore $x$ is uniquely determined by $\frac{w}{r}$, i.e. we have that $\frac{k_t^c}{n_t^c}$ and $\frac{k_t^i}{n_t^i}$ are uniquely determined by $\frac{w}{r}$. 

Moreover, since the same function pin points $\frac{k_t^c}{n_t^c}$ and $\frac{k_t^i}{n_t^i}$, we have $\frac{k_t^c}{n_t^c}=\frac{k_t^i}{n_t^i}$. What about $\frac{k_t}{n_t}$?
\begin{align*}
	 \frac{k_t^c}{n_t^c}& =\frac{k_t^i}{n_t^i}\\
	 k_t^c+k_t^i& =k_t^i\frac{n_t^c}{n_t^i}+k_t^i\\
	 k_t& =k_t^i\left( \frac{n_t^c+n_t^i}{n_t^i}\right) \\
	 \frac{k_t}{n_t}& = \frac{k_t^i}{n_t^i} \\
	 \Rightarrow  \frac{k_t}{n_t}& = \frac{k_t^i}{n_t^i} = \frac{k_t^c}{n_t^c}
\end{align*}

Now, we recall that $p_t^cz_t^cf_k(k_t^c,n_t^c)=p_t^iz_t^if_k(k_t^i,n_t^i)$. Hence,
\begin{align*}
p_t^cz_t^cf_k(k_t^c,n_t^c) &=p_t^iz_t^if_k(k_t^i,n_t^i)\\
\frac{p_t^c}{p_t^i}&=\frac{z_t^i}{z_t^c}\frac{f_k(k_t^i,n_t^i)}{f_k(k_t^c,n_t^c)}\\
\frac{p_t^c}{p_t^i}&=\frac{z_t^i}{z_t^c}\frac{f_k\left( \frac{k_t}{n_t},1\right) }{f_k\left( \frac{k_t}{n_t},1\right)}\\
\frac{p_t^c}{p_t^i}&=\frac{z_t^i}{z_t^c}
\end{align*}

Therefore, we can interpret the relative price of $c$ and $i$ to be equal to the relative total factor productivity between sectors. For example, if the investment sector is more productive, i.e. $z_t^i>z_t^c$, then the price of consumption must be higher than the price of investment to compensate, i.e. $p_t^c>p_t^i$.

Finally, since the input prices and the capital-labor ratios are the same for the aggregate and the individual sectors, it is clear that the economy can be aggregated to one sector. 

We can ask what kind of $z_t$ and $p_t$ would support the aggregate production. Note that the total value of production is given by $ p_t^cz_t^cf(k_t^c,n_t^c) +  p_t^iz_t^if(k_t^i,n_t^i)$.
\begin{align*}
p_tz_tf(k_t,n_t) & = p_t^cz_t^cf(k_t^c,n_t^c) +  p_t^iz_t^if(k_t^i,n_t^i)\\
 & = p_t^cz_t^cf(k_t^c,n_t^c) +  p_t^cz_t^cf(k_t^i,n_t^i) \\
  & = p_t^cz_t^c n_t^c f\left( \frac{k_t^c}{n_t^c},1\right) +  p_t^cz_t^c n_t^if\left( \frac{k_t^i}{n_t^i},1\right)\\
  & = p_t^cz_t^c n_t^c f\left( \frac{k_t}{n_t},1\right) +  p_t^cz_t^c n_t^if\left( \frac{k_t}{n_t},1\right)\\
  & = p_t^cz_t^c  (n_t^c +n_t^i) f\left( \frac{k_t}{n_t},1\right) \\
  & = p_t^cz_t^c  n_t f\left( \frac{k_t}{n_t},1\right) \\
  & = p_t^cz_t^c  f\left(k_t,n_t\right)
\end{align*}

Hence, letting $p_t=p_t^c$ and $z_t=z_t^c$, it is straightforward to see that we can aggregate the two sectors into one. In fact, the household problems becomes
		\begin{align*}
\max_{c_t,k_{t+1},n_{t}} \sum_{t=0}^{\infty}\beta^t u(c_t,1-n_t) \\
\subjectto p_tc_t+\left( \frac{p_tz_t}{z_t^i}\right) i_t &\leq p_tz_t f(k_t,n_t)  \\
i_t &=	k_{t+1}-(1-\delta)k_t  \\
c_t,k_{t+1},n_{t}	& \geq 0
\end{align*}
where $p_t=p_t^c$ and $z_t=z_t^c$.

\section*{Problem 2}
\begin{enumerate}[(1)]
	\item
	For the system to be indeterminate in the steady state, we need to have a situation where the number of equations describing the system is less than the number of variables we are trying to identify.
	
	First, we note that in this particular problem our the set of variables are 
	\begin{enumerate}
		\item Households - $\cbra{c_{it}, h_{it},a_{i,t+1}}$ for $i=1,2$
		\item Aggregate - $\cbra{C_t,N_t,K_T}$
		\item Prices - $\cbra{r_t,w_t,\tau_t}$
	\end{enumerate}
	Note that $T_0$ is set exogenously in this problem so it is not in our set of variables. Hence, in total we need 12 equations in the steady state for the distribution of asset of wealth to be determinate.
	
	
	
	Household problem:
			\begin{align*}
	\max_{c_{it},h_{it},a_{i,t+1}} \sum_{t=0}^{\infty}\beta^t \frac{\left( c_{it}^\alpha(1-h_{it})^{1-\alpha}\right) ^{1-\gamma}}{1-\gamma} \\
	\subjectto c_{it}+a_{i,t+1} & = a_{it}[1+r_t(1-\tau_t)]+ w_t\epsilon_i h_{it} + T^0  \\
	c_{it},h_{it}	& \geq 0
	\end{align*}
	
	FOCs:
	\begin{align*}	
	c_{it}: & \alpha c_{it}^{\alpha-1}(1-h_{it})^{1-\alpha} \left( c_{it}^\alpha(1-h_{it})^{1-\alpha}\right) ^{-\gamma} =  \lambda_t \\
	h_{it}: & (1-\alpha) (1-h_{it})^{-\alpha}  c_{it}^\alpha \left( c_{it}^\alpha(1-h_{it})^{1-\alpha}\right) ^{-\gamma} =  \lambda_t w_t \epsilon_i \\
	a_{i,t+1}: & \lambda_t = \beta [1+r_{t+1}(1-\tau_{t+1})]\lambda_{t+1}\\
	: &c_{it}+a_{i,t+1}  = a_{it}[1+r_t(1-\tau_t)]+ w_t\epsilon_ih_{it} + T^0  \\
	\end{align*}
	
	Equivalently,
	\begin{align*}	
w_t \epsilon_i	\alpha c_{it}^{\alpha-1}(1-h_{it})^{1-\alpha} \left( c_{it}^\alpha(1-h_{it})^{1-\alpha}\right) ^{-\gamma} &=   (1-\alpha) (1-h_{it})^{-\alpha}  c_{it}^\alpha \left( c_{it}^\alpha(1-h_{it})^{1-\alpha}\right) ^{-\gamma}  \\
	\Rightarrow (1-\alpha) c_{it}  &=  \alpha w_t \epsilon_i (1-h_{it}) \\
	c_{it}^{\alpha-1} \left( c_{it}^\alpha(1-h_{it})^{1-\alpha}\right) ^{-\gamma} (1-h_{it})^{-\alpha}  &= \beta R_{t+1}   c_{i,t+1}^{\alpha-1} \left( c_{i,t+1}^\alpha(1-h_{i,t+1})^{1-\alpha}\right) ^{-\gamma} (1-h_{i,t+1})^{-\alpha} \\
		c_{it}^{\alpha-1} \left( c_{it}^\alpha \left(\frac{(1-\alpha)c_{it}}{\alpha w_t \epsilon_i} \right)^{1-\alpha}\right) ^{-\gamma} \left(\frac{(1-\alpha)c_{it}}{\alpha w_t \epsilon_i} \right)^{1-\alpha} &= \beta R_{t+1} c_{i,t+1}^{\alpha-1} \left( c_{i,t+1}^\alpha \left(\frac{(1-\alpha)c_{it}}{\alpha w_t \epsilon_i} \right)^{1-\alpha}\right) ^{-\gamma} \left(\frac{(1-\alpha)c_{i,t+1}}{\alpha w_{t+1} \epsilon_i} \right)^{1-\alpha}\\
		\Rightarrow c_{it}^{-\gamma} \left( \frac{1}{w_t}\right) ^{(1-\alpha)(1-\gamma)} &= \beta R_{t+1} c_{i,t+1}^{-\gamma} \left( \frac{1}{w_{t+1}}\right) ^{(1-\alpha)(1-\gamma)}\\
		c_{i,t+1}= c_{i,t} \left( \beta R_{t+1}\right)^{\frac{1}{\gamma}}  \left( \frac{w_t}{w_{t+1}}\right) ^{\frac{(1-\alpha)(1-\gamma)}{\gamma}}\\
		\Rightarrow c_{i,t+1} &= c_{i,t} \left( \beta [1+r_{t+1}(1-\tau_{t+1})]\right)^{\frac{1}{\gamma}}  \left( \frac{w_t}{w_{t+1}}\right) ^{\frac{(1-\alpha)(1-\gamma)}{\gamma}}
	\end{align*}
	
	Hence, we can summarize the conditions for the household FOCs as
		\begin{align}	(1-\alpha) c_{it}  &=  \alpha w_t \epsilon_i (1-h_{it})\\
		 c_{i,t+1} &= c_{i,t} \left( \beta [1+r_{t+1}(1-\tau_{t+1})]\right)^{\frac{1}{\gamma}}  \left( \frac{w_t}{w_{t+1}}\right) ^{\frac{(1-\alpha)(1-\gamma)}{\gamma}}\\
		c_{it}+a_{i,t+1}  &= a_{it}[1+r_t(1-\tau_t)]+ w_t\epsilon_ih_{it} + T^0 
	\end{align}
	Note that in total this represents $6$ equations since there is two type of people.
 	
 	Recall that for the neoclassical growth model, if the representative firm has CRS technology, then the FOCs of its problem are such that the marginal product of the inputs are equal to their respective marginal prices, i.e.
 			\begin{align}	
 			F_K(K_t,N_t) &= r_t+\delta\\
 			F_N(K_t,N_t) &= w_t
 	\end{align}
 	
 	
 	Now, the government problem is quite an easy one as its budget constraint is given and therefore we have
 	 			\begin{align}	
 T^0 = r_tK_t\tau_t
 	\end{align}
 	
 	Thus, at this point we have 9 equations. The remaining 3 are quite straightforward since they are simply the aggregation equations using $\mu_i$, i.e.
 			\begin{align}	
 	K_t &= \mu_1a_{1t} + \mu_2a_{2t}\\
 	N_t &= \mu_1\epsilon_1h_{1t} + \mu_2\epsilon_2h_{2t}\\
 	C_t &= \mu_1c_{1t} + \mu_2c_{2t}
 	\end{align}
 	
 	Hence, we have 12 equations and 12 variables! Thus, given the initial starting value we can solve for the sequences of variables. 
 	
 	What about the steady state? Note that in the steady state, \[c_{i,t+1} = c_{i,t} \left( \beta [1+r_{t+1}(1-\tau_{t+1})]\right)^{\frac{1}{\gamma}}  \left( \frac{w_t}{w_{t+1}}\right) ^{\frac{(1-\alpha)(1-\gamma)}{\gamma}}\] becomes
 	\[
 	1 =  \left( \beta [1+r(1-\tau)]\right)^{\frac{1}{\gamma}}
 	\]
 	
 	Therefore, while $c_{i,t+1} = c_{i,t} \left( \beta [1+r_{t+1}(1-\tau_{t+1})]\right)^{\frac{1}{\gamma}}  \left( \frac{w_t}{w_{t+1}}\right) ^{\frac{(1-\alpha)(1-\gamma)}{\gamma}}$ are two distinct equations for $i=1,2$, they collapse to the same equation in steady state. Hence, we now have 11 equations and the system becomes indeterminate in steady state.
 	
 		 Let $K$ be the capital stock in steady state, then the set of possible steady state for $\cbra{a_i}$ are any possible combinations on the line $K = \mu_1a_{1} + \mu_2a_{2}$. Hence, the final-steady state is not uniquely determined.
 	
 	
	\item
	
	To show that the economy has a representative agent formulation, we can show that the demand functions admits a linear in individual wealth.
	
	First, what is wealth in this economy at $t$ for agent $i$? For this, we need to reformulate our budget constraint. 
	
	By iterating our budget constraint, we get
	\[
	\sum_{j=t}^T\left(  \prod_{j=t}^{T}\frac{1+(1-r_t)\tau_t}{1+(1-r_j)\tau_j}\right) c_{j} = \sum_{j=t}^T\left(  \prod_{j=t}^{T}\frac{1+(1-r_t)\tau_t}{1+(1-r_j)\tau_j}\right) (w_j\epsilon_ih_{ij}+T) + \left(  \prod_{j=t}^{T}\frac{1+(1-r_t)\tau_t}{1+(1-r_j)\tau_j}\right) a_{iT}
	\]
	
	Let $\left(  \prod_{j=t}^{T}\frac{1+(1-r_t)\tau_t}{1+(1-r_j)\tau_j}\right) = q^t_j$. Since we have a non-ponzi scheme conditions, we have
\begin{align*}
	\sum_{j=t}^\infty q^t_j c_{j} &= \sum_{j=t}^\infty q^t_j (w_j\epsilon_ih_{ij}+T)\\
	\Rightarrow \sum_{j=t}^\infty q^t_j c_{j}+ \sum_{j=t}^\infty q^t_j w_j\epsilon_i(1-h_{ij}) &= \sum_{j=t}^\infty q^t_j (w_j\epsilon_i+T)
\end{align*}
On the left side we have total expenditure on leisure and consumption. Hence, on the right side, we have the present value of wealth at time $t$. Let $\sum_{j=t}^Tq^t_j (w_j\epsilon_i+T) = m_{it}$ for simplicity.

Now, we ask: can we write $(1-h_{it})$ and $c_{it}$ as linear function $m_it$? First, recall the FOCs we derived in part 1:
		\begin{align*}	(1-\alpha) c_{it}  &=  \alpha w_t \epsilon_i (1-h_{it})\\
c_{i,t+1} &= c_{i,t} \left( \beta [1+r_{t+1}(1-\tau_{t+1})]\right)^{\frac{1}{\gamma}}  \left( \frac{w_t}{w_{t+1}}\right) ^{\frac{(1-\alpha)(1-\gamma)}{\gamma}}
\end{align*}

First, let $j>t$, then 
	\begin{align*}	c_{i,t+1} &= c_{i,t} \left( \beta [1+r_{t+1}(1-\tau_{t+1})]\right)^{\frac{1}{\gamma}}  \left( \frac{w_t}{w_{t+1}}\right) ^{\frac{(1-\alpha)(1-\gamma)}{\gamma}}\\
c_{i,j} &= c_{i,t}  \prod_{s=t}^{j-1}\left( \beta [1+r_{s+1}(1-\tau_{s+1})]\right)^{\frac{1}{\gamma}} \left( \frac{w_s}{w_{s+1}}\right) ^{\frac{(1-\alpha)(1-\gamma)}{\gamma}}\\
\Rightarrow c_{i,j}&= c_{i,t}  \beta^{\frac{j-t}{\gamma}} \frac{1}{q_j^t} \left( \frac{w_t}{w_{j}}\right) ^{\frac{(1-\alpha)(1-\gamma)}{\gamma}}
\end{align*}


Now, $(1-\alpha) c_{it}  =  \alpha w_t \epsilon_i (1-h_{it})$ yields that
\begin{align*}
\sum_{j=t}^\infty q^t_j c_{j} &= \alpha m_{it}\\
\sum_{j=t}^\infty  q^t_j w_j\epsilon_i(1-h_{ij}) &= (1-\alpha)m_{it}
\end{align*}

Combining all of this together, we get
	\begin{align*}
	\sum_{j=t}^Tq^t_j c_{j} &= \alpha m_{it}\\
	\sum_{j=t}^T c_{i,t}  \beta^{\frac{j-t}{\gamma}} \left( \frac{w_t}{w_{j}}\right) ^{\frac{(1-\alpha)(1-\gamma)}{\gamma}} &= \alpha m_{it}\\
	c_{i,t} &= \underbrace{\left( \frac{\alpha}{\sum_{j=t}^T   \beta^{\frac{j-t}{\gamma}} \left( \frac{w_t}{w_{j}}\right)^{\frac{(1-\alpha)(1-\gamma)}{\gamma}}}\right) }_{A^t} m_{it}\\
	\Rightarrow c_{i,t}& = A^t m_it
	\Rightarrow (1-h_{i,t})\epsilon_i = \underbrace{\frac{(1-\alpha)}{\alpha w_t } A^t}_{B^{t}} m_it
	\end{align*}
	
	Since $A^{t}$ and $B^{t}$ do not depend on $i$, we have that the $(1-h_{i,t})\epsilon_i$ and $c_{i,t}$ are linear in wealth $m_{it}$ and therefore the economy admits a representative agent formulation.
	
	\item 
	

What happens when $T^1>T^0$?. 

First, there's a wealth effect since agents have more disposable income thanks to the increase in $T$. 

But, an increase in government spending implies that we need to finance it through some change in tax. This create a substitution effect. 

Now, usually more government spending implies that we need an increase in tax to fund it. But it could well be the fact that we are a situation where tax are simply too high which crowds out the economy and that a decrease in tax would return a higher amount of government spending. To see how this could be the case, let $\tau_t=1$. Then, there is no incentive to save and the total revenue for governments are equal to $0$.

Therefore, the effect on capital stock is unclear.

For the sake of discussion, let's assume that any $T^1>T^0$ must be finance with a tax increase. First, note that since $1 =  \left( \beta [1+r(1-\tau)]\right)^{\frac{1}{\gamma}}$ any movement in $\tau$ must be match with a change in the same direction from $r$. Hence, a tax increase would imply a increase in $r$.

Now, since $F_K(\frac{K}{N},1)=r+\delta$, this increase in $r$ would be match by a decrease in $\frac{K}{N}$. This in turns tells us that $F_N(\frac{K}{N},1)=w$ implies a decrease in $w$. But a decrease in $w$ would imply a decrease in return to working for agents, hence they would opt to decrease their supply of $\epsilon_i h{it}$. Finally, this decrease in labor supply implies a decrease in $N$ and since $\frac{K}{N}$ is decreasing, we need $K$ to decrease as well. Therefore, in this scenario, $K$ will drop.

Recalling that any possible combinations on the line $K = \mu_1a_{1} + \mu_2a_{2}$ are a possible steady state, we draw the dynamics of an increase in $T^1>T^0$ in Fig. 1 by showing the transition of the two lines and the transition of one of the possible steady state.	
	\begin{figure}[H]
	
	\begin{center}
		\begin{tikzpicture}
		\draw[->] (0,0)--(0,8) node[left] {$a_2$};
		\draw[->] (0,0)--(8,0) node[below] {$a_1$};
		\draw[thick] (0,6)--(6,0) ;
		\draw[thick, ->] (3,3)--(2,2);
		\draw[thick] (0,4)--(4,0) ;
		
		\coordinate[label = right : $\text{Steady-state}^0$] (A) at (3.2,3.2);
		\coordinate[label = left : $\text{Steady-state}^1$] (B) at (2,1.8);
		\coordinate[label = left : $\frac{K^0}{\mu_2}$] (C) at (0,6);
		\coordinate[label = below : $\frac{K^0}{\mu_1}$] (D) at (6,0);
		\coordinate[label = left : $\frac{K^1}{\mu_2}$] (E) at (0,4);
		\coordinate[label = below : $\frac{K^1}{\mu_1}$] (F) at (4,0);
		\end{tikzpicture}
	\end{center}
\end{figure}

Note that since, we are given a starting steady-state, the dynamic equations are enough to completely describe the convergence when $T^1>T^0$, i.e. we know the sequence of $a_{it}$ if we have $a_{i0}$. But again, even with the new $T^1$ wealth distribution steady state is still indeterminate. 
	
\end{enumerate}


\section*{Problem 3}
\begin{enumerate}[(1)]
	\item
	
	\begin{definition}
		A competitive Arrow-Debreu equilibrium is a set of prices $\seq{\hat{p}_t}_{t=0}^{\infty}$ and allocations $(\seq{\hat{c}^i_t}_{t=0}^{\infty})$ such that
		\begin{enumerate}
			\item Given $\seq{\hat{p}_t}_{t=0}^{\infty}$, for $i = 1, 2$, $(\seq{\hat{c}^i_t}_{t=0}^{\infty})$ solves
			\[
			\begin{split}
			\max_{\seq{\hat{c}^i_t}_{t=0}^{\infty}} \sum_{t=0}^{\infty}\beta^t \log(c_t^i)\\
			\subjectto \sum_{t=0}^{\infty}\hat{p}_tc_t^i\leq \sum_{t=0}^{\infty}\hat{p}_te_t^i\\
			c_{t}^i\geq 0 \text{ for all }t
			\end{split}
			\]
			\item Market clearing
			\[\hat{c}_t^1+\hat{c}_t^2=e_t^1+e_t^2 \text{ for all }t \]
		\end{enumerate}
	\end{definition}


	Note that in part 4 of this problem set, we show that there exists a set of weight for the planners problem where the solution coincide with the competitive ADE. Hence, the competitive ADE is Pareto optimal. But even without jumping the gun, it is straightforward to see that the conditions for the first Welfare Theorem hold, i.e.
	\begin{enumerate}[(i)]
		\item Price-taking $\Rightarrow$ Comes from the definition of the problem. 
		\item Complete Markets $\Rightarrow$ Only one state of nature, hence the markets are complete. 
		\item Local Non-satiation $\Rightarrow$ $u(\cdot)$ is strictly increasing and preferences are locally non-satiated.
	\end{enumerate}
	
	
	
	\item
	
	Note that the Langrangian of the household problem is given by:
	\[
	\La = \sum_{t=0}^{\infty} \beta^t \log(c^i_t) + \lambda \sum_{t=0}^{\infty}(\hat{p}_te^i_t-\hat{p}_tc_t^i)
	\] 
	FOCs:
	\[
\begin{split}
	& \frac{\beta^t }{c^i_t} = \lambda_t \hat{p}_t\\
	\Rightarrow & \hat{p}_{t+1}c_t^i = \beta\hat{p}_tc_t^i
	\end{split}
	\]
	
	Adding the previous result with the market clearing conditions yields
	\[ \begin{split}
	\hat{p}_{t+1}(c_{t+1}^1+c_{t+1}^2) & = \beta \hat{p}_t(c_{t}^1+c_{t}^2)\\
	\hat{p}_{t+1}(e_{t+1}^1+e_{t+1}^2) & = \beta \hat{p}_t(e_{t}^1+e_{t}^2)\\
	\Rightarrow \hat{p}_{t+1}=\beta \hat{p}_t
	\end{split} \]
	
	WLOG let $\hat{p}_0=1$, then we have $\hat{p}_t=\beta^t$. This yields that $c_{t+1}^i=c_t^i = c_0^i$. 
	
	Additionally, note that the value of the endowment streams give us
	\[
	\begin{split}
	\sum_{t=0}^{\infty}\hat{p}_te_t^1 &= 2 \sum_{t=0}^{\infty}\beta^{2t}= \frac{2}{1-\beta^2}\\
	\sum_{t=0}^{\infty}\hat{p}_te_t^2 &= 2\beta \sum_{t=0}^{\infty}\beta^{2t}= \frac{2\beta}{1-\beta^2}
	\end{split}
	\]
	
	Hence, we have that
	\[
	\hat{c}^1_t = \frac{2}{1+\beta}\qquad \hat{c}^2_t = \frac{2\beta}{1+\beta}
	\]
	and therefore, for all $t$,
	\[\hat{c}^1_t > \hat{c}^2_t\]

	
	
	\item
	Social Planner problem:
	\[
	\begin{split}
	&\max_{c^1,c^2}\sum_{t=0}^{\infty}\beta^t \bra{\alpha \log(c_t^1)+(1-\alpha)log(c_t^2)}\\
	& c_t^i\geq 0, \forall i, \forall t\\
	& c_t^1+c_t^2=e_t^1+e_t^2=2, \forall t
	\end{split}
	\]
	
	Attach $\frac{\theta_t}{2}$ as the multipliers for the resource constraints in the Lagrangian
	
	FOCs:
	\[
	\begin{split}
	&\frac{\alpha \beta^t}{c_t^1}=\frac{\theta_t}{2}\\
	&\frac{(1-\alpha) \beta^t}{c_t^2}=\frac{\theta_t}{2}\\
	\end{split}
	\]
	and therefore
	\[
	c_t^1 = \frac{\alpha}{1-\alpha}c_t^2
	\]
	
	Combining with the resource constraints, we get
	\[
	\begin{split}
	c_t^1(\alpha)&=2\alpha\\
	c_t^2(\alpha)&=2(1-\alpha)\\
	\theta_t&=\beta^t
	\end{split}
	\]
	
	Note that the $\theta_t$ are equivalent to the Arrow-Debreu prices.
	
	\item
	For any $\alpha$, we can write the following transfer function
	\[
	t^i(\alpha)=\sum_t \hat{p}_t\bra{c_t^i(\alpha)-e_t^i}
	\]
	
	Hence, $t^i(\alpha)$ represent the present value needed in extra to the endowment process of $i$ to support $c_t^i(\alpha)$.
	
	In fact, we have
	\[
	\begin{split}
	t^1(\alpha)&=\sum_t \hat{p}_t\bra{c_t^1(\alpha)-e_t^1}= \sum_t \beta^t\bra{2\alpha-e_t^1}= \frac{2\alpha}{1-\beta}-\frac{2}{1-\beta^2}\\
	t^2(\alpha)&=\sum_t \hat{p}_t\bra{c_t^2(\alpha)-e_t^2}=\sum_t \beta^t\bra{2(1-\alpha)-e_t^2}=\frac{2(1-\alpha)}{1-\beta}-\frac{2\beta}{1-\beta^2}
	\end{split}
	\]
	
	To find the specific $\alpha^*$ that brings us back to the ADE, we need to look at the case where $t^1(\alpha)=t^4(\alpha)=0$, i.e. there is no transfer. 
	
	It is straightforward to see that $\alpha^*=\frac{1}{1+\beta}$. Additionally, for that $\alpha^*$, the consumption sequences are
	\[
	\begin{split}
	c_t^1(\alpha)&=\frac{2}{1+\beta}\\
	c_t^2(\alpha)&=\frac{2\beta}{1+\beta}
	\end{split}
	\]	
	which is what we got when we solved the ADE.	
	

\end{enumerate}

\end{document}
