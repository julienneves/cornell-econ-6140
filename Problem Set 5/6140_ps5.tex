% Class Notes Template
\documentclass[12pt]{article}
\usepackage[margin=1in]{geometry} 
\usepackage[utf8]{inputenc}

% Packages
\usepackage[french, english]{babel}
\usepackage{amsmath, amsthm, amssymb ,amsfonts, graphics, tikz, float, enumerate, graphicx}
\usepackage{listings}
\usepackage{color} %red, green, blue, yellow, cyan, magenta, black, white
\definecolor{mygreen}{RGB}{28,172,0} % color values Red, Green, Blue
\definecolor{mylilas}{RGB}{170,55,241}

\lstset{language=Matlab,%
	%basicstyle=\color{red},
	breaklines=true,%
	morekeywords={matlab2tikz},
	keywordstyle=\color{blue},%
	morekeywords=[2]{1}, keywordstyle=[2]{\color{black}},
	identifierstyle=\color{black},%
	stringstyle=\color{mylilas},
	commentstyle=\color{mygreen},%
	showstringspaces=false,%without this there will be a symbol in the places where there is a space
	numbers=left,%
	numberstyle={\tiny \color{black}},% size of the numbers
	numbersep=9pt, % this defines how far the numbers are from the text
	emph=[1]{for,end,break},emphstyle=[1]\color{blue}, %some words to emphasise
	%emph=[2]{word1,word2}, emphstyle=[2]{style},    
}

% Title
\title{ECON 6140 - Problem Set \# 5}
\date{\today}
\author{Julien Manuel Neves}

% Use these for theorems, lemmas, proofs, etc.
\newtheorem{theorem}{Theorem}
\newtheorem{corollary}[theorem]{Corollary}
\newtheorem{lemma}[theorem]{Lemma}
\newtheorem{observation}[theorem]{Observation}
\newtheorem{proposition}[theorem]{Proposition}
\newtheorem{definition}[theorem]{Definition}
\newtheorem{claim}[theorem]{Claim}
\newtheorem{fact}[theorem]{Fact}
\newtheorem{assumption}[theorem]{Assumption}
\newtheorem{problem}[theorem]{Problem}
\newtheorem{set-up}[theorem]{Set-up}
\newtheorem{example}[theorem]{Example}
\newtheorem{remark}[theorem]{Remark}
\newtheorem{axiom}[theorem]{Axiom}

% Usefuls Macros
\newcommand{\field}[1]{\mathbb{#1}}
\newcommand{\N}{\field{N}} % natural numbers
\newcommand{\R}{\field{R}} % real numbers
\newcommand{\Z}{\field{Z}} % integers
\newcommand\F{\mathcal{F}}
\newcommand\B{\mathbb{B}}
\renewcommand{\Re}{\R} % reals
\newcommand{\Rn}[1]{\mathbb{R}^{#1}}
\newcommand{\1}{{\bf 1}} % vector of all 1's
\newcommand{\I}[1]{\mathbb{I}_{\left\{#1\right\}}} % indicator function
\newcommand{\La}{\mathscr{L}}
% \newcommand{\tends}{{\rightarrow}} % arrow for limits
% \newcommand{\ra}{{\rightarrow}} % abbreviation for right arrow
% \newcommand{\subjectto}{\mbox{\rm subject to}} % subject to

%% math operators
\DeclareMathOperator*{\argmin}{arg\,min}
\DeclareMathOperator*{\argmax}{arg\,max}
\DeclareMathOperator*{\maximize}{maximize}
\DeclareMathOperator*{\minimize}{minimize}
\DeclareMathOperator{\E}{\mathsf{E}} % expectation
\newcommand{\Ex}[1]{\E\left\{#1\right\}} % expectation with brackets
\DeclareMathOperator{\pr}{\mathsf{P}} % probability
\newcommand{\prob}[1]{\pr\left\{#1\right\}}
\DeclareMathOperator{\subjectto}{{s.t.\ }} % subject to
\newcommand{\norm}[1]{\left\|#1\right\|}
\newcommand{\card}[1]{\left|#1\right|}

% Extra stuff
\newcommand\seq[1]{\{ #1 \}}
\newcommand{\inp}[2]{\langle #1, #2 \rangle}

\newcommand{\inv}{^{-1}}

\newcommand{\pa}[1]{\left(#1\right)}
\newcommand{\bra}[1]{\left[#1\right]}
\newcommand{\cbra}[1]{\left\{ #1 \right\}}

\newcommand{\pfrac}[2]{\pa{\frac{#1}{#2}}}
\newcommand{\bfrac}[2]{\bra{\frac{#1}{#2}}}

\newcommand{\mat}[1]{\begin{matrix}#1\end{matrix}}
\newcommand{\pmat}[1]{\pa{\mat{#1}}}
\newcommand{\bmat}[1]{\bra{\mat{#1}}}

\begin{document}

\maketitle

\section*{Endogenous growth and government spending}

\begin{enumerate}[(1)]
	\item 
	The Hamiltonian for this problem is given by
	\begin{align*}
&\mathcal{H}(c(t),k(t),\lambda(t)) = e^{-\rho t}u(c(t)) +\lambda(t)[Ak(t)-c(t)-\delta k(t)]
\end{align*}

Hence, we get the following sufficient conditions
\begin{align*}
e^{-\rho t}u'(c(t)) & = \lambda(t)\\
\dot{\lambda}(t) & = -\lambda(t) [A-\delta]\\
\lim_{t\to \infty} \lambda(t) k(t) & = 0\\
	 \dot{k}(t) & =Ak(t)-c(t)-\delta k(t)
\end{align*}

First, we need to take the derivative of the first condition with respect to $t$
\begin{align*}
-\rho e^{-\rho t}u'(c(t)) + e^{-\rho t}u''(c(t))\dot{c}(t) & = \dot{\lambda}(t)\\
-\rho e^{-\rho t}c^{-\sigma}- e^{-\rho t} \sigma c^{-\sigma-1}\dot{c}(t) & = \dot{\lambda}(t)
\end{align*}

Hence,
\begin{align*}
\dot{\lambda}(t) & = -\lambda(t) [A-\delta]\\
-\rho e^{-\rho t}c^{-\sigma}- e^{-\rho t} \sigma c^{-\sigma-1}\dot{c}(t) & = -e^{-\rho t}c^{-\sigma}  [A-\delta]\\
 \frac{\dot{c}(t)}{c(t)} & = \frac{1}{\sigma} [A-\delta-\rho]
\end{align*}

Since the growth rate of $\frac{\dot{c}(t)}{c(t)}$ is constant, we are a balanced growth path. Moreover, we need $k$ to grow at the same rate. This is straightforward if you look at the following condition
\begin{align*}
\frac{\dot{k}(t)}{k(t)} & =A-\frac{c(t)}{k(t)}-\delta
\end{align*}
	\item 
	
	Solving for $\frac{\dot{c}(t)}{c(t)} = \frac{1}{\sigma} [A-\delta-\rho]$ yields
	\[
	c(t) = e^{\frac{1}{\sigma} [A-\delta-\rho]}c(0)
	\]
	
	If we plug this back in the utility function, we get
	\begin{align*}
	U & = \int_{0}^{\infty}e^{-\rho t} \frac{e^{\frac{(1-\sigma)}{\sigma} [A-\delta-\rho]t}}{1-\sigma}c(0)^{(1-\sigma)} dt \\
	& = \int_{0}^{\infty}\frac{e^{\frac{(1-\sigma)}{\sigma} [A-\delta-\frac{\rho}{1-\sigma}]t}}{1-\sigma}c(0)^{(1-\sigma)} dt 
	\end{align*}
	
	Thus, for $U$ to be bounded, we need
	\[
	\frac{(1-\sigma)}{\sigma} \left[ A-\delta-\frac{\rho}{1-\sigma}\right] <0
	\]
	\item 
	
	Note that $MPK=\phi\left(\frac{g}{k}\right) - \frac{g}{k}\phi'\left(\frac{g}{k}\right) $
	\begin{align*}
		\eta & = -\frac{\partial y}{\partial g} \cdot \frac{g}{y}\\
		& = -\phi'\left(\frac{g}{k}\right) \cdot \frac{g}{k \phi\left(\frac{g}{k}\right)}\\
		& =  \frac{MPK-\phi\left(\frac{g}{k}\right)}{ \phi\left(\frac{g}{k}\right)}\\
		& =  \frac{MPK}{ \phi\left(\frac{g}{k}\right)}-1\\
		\Rightarrow MPK &= \phi\left(\frac{g}{k}\right)(1+\eta)
	\end{align*}
	
	\item 
		The Hamiltonian for this problem is given by
	\begin{align*}
	&\mathcal{H}(c(t),k(t),\lambda(t)) = e^{-\rho t}u(c(t)) +\lambda(t)[f(k(t),g)-g-c(t)-\delta k(t)]
	\end{align*}
	
	The new sufficient conditions are the following
	\begin{align*}
	e^{-\rho t}u'(c(t)) & = \lambda(t)\\
	\dot{\lambda}(t) & = -\lambda(t) [MPK-g-\delta]\\
	\lim_{t\to \infty} \lambda(t) k(t) & = 0\\
	\dot{k}(t) & =Ak(t)-c(t)-\delta k(t)
	\end{align*}
	
	First, we need to take the derivative of the first condition with respect to $t$
	\begin{align*}
	-\rho e^{-\rho t}u'(c(t)) + e^{-\rho t}u''(c(t))\dot{c}(t) & = \dot{\lambda}(t)\\
	-\rho e^{-\rho t}c^{-\sigma}- e^{-\rho t} \sigma c^{-\sigma-1}\dot{c}(t) & = \dot{\lambda}(t)
	\end{align*}
	
	Hence,
	\begin{align*}
	\dot{\lambda}(t) & = -\lambda(t) [MPK-g-\delta]\\
	-\rho e^{-\rho t}c^{-\sigma}- e^{-\rho t} \sigma c^{-\sigma-1}\dot{c}(t) & = -e^{-\rho t}c^{-\sigma}  [MPK-g-\delta]\\
	\frac{\dot{c}(t)}{c(t)} & = \frac{1}{\sigma} [MPK-g-\delta]\\
	\frac{\dot{c}(t)}{c(t)} & = \frac{1}{\sigma} \left[ \phi\left(\frac{g}{k}\right)(1+\eta)-g-\delta\right] 
	\end{align*}
	
	Finally,
		\begin{align*}
	\frac{\partial \left( \frac{\dot{c}(t)}{c(t)}\right)}{\partial g} & = \frac{1}{\sigma} \left[ \frac{1}{k}\phi'\left(\frac{g}{k}\right)(1+\eta)-1\right] 
	\end{align*}
	
	Note that $\frac{1}{k}\phi'\left(\frac{g}{k}\right)(1+\eta)>0$. Therefore, the change in growth rate will depend on if the increase in $g$ is such that $\phi'\left(\frac{g}{k}\right)(1+\eta) \geq k$ or $\phi'\left(\frac{g}{k}\right)(1+\eta) \leq k$. 
\end{enumerate}
	
	
\section*{The value of life}
	
\begin{enumerate}[(1)]
	\item 
	
	The problem can be summarize by the following value function
	\begin{align*}
V(c)= \max_{s\in \cbra{\text{research, stop}}} U^s
	\end{align*}
	where $U^S = \left\lbrace \mat{(1-\pi)u(c(1+g)) & \text{ if }s=\text{research}\\u(c) & \text{ if }s=\text{stop}} \right. $.
	
	Thus, for the agent to choose "research", we need the following condition
	\[
	(1-\pi)u(c(1+g))\geq u(c)
	\]
	
	\item
	
	First order Taylor expansion and let $L$ be the value of life.
	\begin{align*}
		u(c_1) & = u(c)+u'(c)(c_1-c)\\
		& = u(c)+u'(c)gc\\
		\Rightarrow \frac{u(c_1)}{u'(c)} & = \frac{u(c)}{u'(c)} +gc\\
		u(c_1) & = \frac{u(c)}{u'(c)}c^{-\sigma} +gc^{1-\sigma}\\
		u(c_1) & = Lc^{-\sigma} +gc^{1-\sigma}
	\end{align*}
	
	Hence, the previous optimality condition reduce to
		\begin{align*}
(1-\pi)u(c(1+g)) &\geq u(c)\\
(1-\pi)(Lc^{-\sigma} +gc^{1-\sigma}) &\geq u(c)\\
(1-\pi)L + (1-\pi)gc &\geq \frac{u(c)}{c^{-\sigma}}\\
(1-\pi)L + (1-\pi)gc &\geq L\\
(1-\pi)gc &\geq \pi L \\
\frac{(1-\pi)}{\pi} gc &\geq L
	\end{align*}
	 
	 This can be interpreted as the following: the cost of dying $\pi L$ needs to be less than expected gain in consumption $gc$ for the agent to chose research.
	\item 
	
	Recall that 
			\begin{align*}
	\frac{(1-\pi)}{\pi} gc &\geq L\\
	g &\geq  \frac{\pi}{(1-\pi)} \frac{u(c)}{u'(c)c}\\
	g &\geq  \frac{\pi}{(1-\pi)} \left( \frac{\bar{u}}{c^{1-\sigma}} + \frac{1}{1-\sigma}\right)
	\end{align*}
	
	Note that while the growth rate $g$ stays the same for any change in $\sigma$, any change in $\sigma$ could potential affect the decision of the agent to do research or not, i.e. grow.
	
	In fact, what's of interest for us is how $\frac{\pi}{(1-\pi)} \left( \frac{\bar{u}}{c^{1-\sigma}} + \frac{1}{1-\sigma}\right)$ changes with respect to $\sigma$
			\begin{align*}
\frac{\partial \left( \frac{\pi}{(1-\pi)} \frac{u(c)}{u'(c)c}\right) }{\partial \sigma} & =\frac{\pi}{(1-\pi)} \left( \frac{\bar{u}\log(c)}{c^{1-\sigma}} + \frac{1}{(1-\sigma)^2}\right)\\
& =\frac{\pi}{(1-\pi)} \left( \frac{\bar{u}\log(c)}{c^{1-\sigma}} -\frac{\bar{u}}{(1-\sigma)c^{1-\sigma}} + \frac{L}{c(1-\sigma)}\right)\\
& =\frac{\pi}{(1-\pi)} \left( \frac{\bar{u}((1-\sigma)\log(c)-1)}{(1-\sigma)c^{1-\sigma}} + \frac{L}{c(1-\sigma)}\right)\\
& =\frac{\pi}{(1-\pi)} \frac{1}{(1-\sigma)c^{1-\sigma}}\left( \bar{u}((1-\sigma)\log(c)-1) +Lc^{-\sigma}\right)\\
\end{align*}
	Note that if we assume that $c\geq 1$, then $\frac{\partial \left( \frac{\pi}{(1-\pi)} \frac{u(c)}{u'(c)c}\right) }{\partial \sigma}>0$. If $c<1$, then $\log(c)<0$ and the sign of $\frac{\partial \left( \frac{\pi}{(1-\pi)} \frac{u(c)}{u'(c)c}\right) }{\partial \sigma}$ is less straightforward. I'll assume that $c\geq 1$ for this example, i.e. $\frac{\partial \left( \frac{\pi}{(1-\pi)} \frac{u(c)}{u'(c)c}\right) }{\partial \sigma}>0$. 
	
	Thus, an increase in $\sigma$, implies that the right hand side of the optimality condition also increase, i.e. less likely that $g \geq  \frac{\pi}{(1-\pi)} \left( \frac{\bar{u}}{c^{1-\sigma}} + \frac{1}{1-\sigma}\right) $. In other words, the less risk averse/ the higher the elasticity of inter-temporal substitution, the less likely the agent is to under take research.
	
	Moreover, it is clear that the same can be said for $L$ and $\bar{u}$. The more they increase, the more the agent values being alive, the less likely to take the chance of dying, and the faster $\frac{\partial \left( \frac{\pi}{(1-\pi)} \frac{u(c)}{u'(c)c}\right) }{\partial \sigma}$ increase.
	
	\item 
	
	For sake of clarity, we enumerate all constraints faced by the social planner, i.e.
	\begin{align*}
		&\max_{c_t} \int_{0}^{\infty}e^{-\rho t}u(c_t)M_tdt\\
		\subjectto & C_t = \left( \int_{0}^{A_t}x_{it}^{\frac{1}{1+\alpha}}di\right) ^{1+\alpha}\\
		& H_t = \left( \int_{0}^{B_t}z_{it}^{\frac{1}{1+\alpha}}di\right) ^{1+\alpha}\\
		& \int_{0}^{A_t}x_{it}dt+  \int_{0}^{B_t}z_{it}dt = L_{at}+ L_{bt}=L_t\\
		& \dot{A}_t = S_{at}^\lambda A_t^\phi\\
		& \dot{B}_t = S_{bt}^\lambda B_t^\phi\\	
		& \dot{M}_t = -\delta_tM_t\\
		& \dot{N}_t = \bar{n}N_t\\
		& N_t = S_t+L_t\\
		& M_0 = 1, \delta_t = -h^\beta, h_t = \frac{H_t}{N_t}
	\end{align*}
	
	First note that for the social planner the optimal solution is such that $x_{it}=x_t$ and $z_{it}=z_t$ for all $i$.
	
	This yields the following 
		\begin{align*}
		 L_{at} &=  \int_{0}^{A_t}x_{it}dt\\
		 &=  \int_{0}^{A_t}x_{t}dt\\
		 &=  A_tx_{t}\\
		 L_{bt} &= \int_{0}^{B_t}z_{it}dt \\
		 &= \int_{0}^{B_t}z_{t}dt \\
		 & = B_t z_{t}\\
		 C_t & = \left( \int_{0}^{A_t}x_{it}^{\frac{1}{1+\alpha}}di\right) ^{1+\alpha}\\
		 & = \left(x_{t}^{\frac{1}{1+\alpha}} \int_{0}^{A_t}1di\right) ^{1+\alpha}\\
		 & = {A_t}^{1+\alpha}x_{t}\\
		 & = {A_t}^{\alpha} L_{at}\\
		 H_t &= \left( \int_{0}^{B_t}z_{it}^{\frac{1}{1+\alpha}}di\right) ^{1+\alpha}\\
		 &= \left( \int_{0}^{B_t}z_{t}^{\frac{1}{1+\alpha}}di\right) ^{1+\alpha}\\
		 		 & = {B_t}^{1+\alpha}z_{t}\\
		 & = {B_t}^{\alpha} L_{bt}
	\end{align*}
	
	Now, let $s_t= \frac{S_{at}}{S_t}$, $l_t=\frac{L_{at}}{L_t}$, and $\sigma_t = \frac{S_t}{N_t}$.
	
	This implies the following
		\begin{align*}
& c_t = \frac{ {A_t}^{\alpha} L_{at}}{N_t} ={A_t}^{\alpha} (l_t)(1-\sigma_t)\\
	& h_t = \frac{ {B_t}^{\alpha} L_{bt}}{N_t} ={B_t}^{\alpha} (1-l_t)(1-\sigma_t) \\
	& \dot{A}_t = S_{at}^\lambda A_t^\phi = (s_t\sigma_tN_t)^\lambda A_t^\phi\\
	& \dot{B}_t = S_{bt}^\lambda B_t^\phi = ((1-s_t)\sigma_tN_t)^\lambda B_t^\phi\\
	& \dot{M}_t = -({B_t}^{\alpha} (1-l_t)(1-\sigma_t))^{-\beta} M_t\\
	& \dot{N}_t = \bar{n}N_t\\
	& M_0 = 1
	\end{align*}
	
	Hence, the social planner problem is given by
			\begin{align*}
			&\max_{c_t} \int_{0}^{\infty}e^{-\rho t}u\left( {A_t}^{\alpha} (l_t)(1-\sigma_t)\right) M_tdt\\
	\subjectto & \dot{A}_t = S_{at}^\lambda A_t^\phi = (s_t\sigma_tN_t)^\lambda A_t^\phi\\
	& \dot{B}_t = S_{bt}^\lambda B_t^\phi = ((1-s_t)\sigma_tN_t)^\lambda B_t^\phi\\
	& \dot{M}_t = -({B_t}^{\alpha} (1-l_t)(1-\sigma_t)^{-\beta})M_t\\
	& \dot{N}_t = \bar{n}N_t
	\end{align*}
	where $M_0 = 1$.
	
	Finally, this yields this Hamiltonian
			\begin{align*}
\mathcal{H} & =\left\lbrace e^{-\rho t}u\left( {A_t}^{\alpha} (l_t)(1-\sigma_t)\right) M_t \right.  + \eta_t\left[ (s_t\sigma_tN_t)^\lambda A_t^\phi\right]  + \mu_t\left[ ((1-s_t)\sigma_tN_t)^\lambda B_t^\phi\right] \\ &+ \nu_t\left[  -({B_t}^{\alpha} (1-l_t)(1-\sigma_t))^{-\beta} M_t\right]  \left. + \theta_t  \bar{n}N_t\right\rbrace 
\end{align*}
	
	\item 
	
	See previous part.
	
		\item 
		
		In balanced growth path, we have that 
		\[
		\frac{\dot{A}}{A} = \gamma_A
		\]
		for some constant $\gamma_A$.
		
		Hence,
		\begin{align*}
		 \gamma_A& = 	\frac{\dot{A}}{A} \\
		 &=	\frac{S_{at}^\lambda A_t^\phi }{A_t}\\
		 &=	\frac{S_{at}^\lambda }{A_t^{\1-\phi}}
		\end{align*}
		
		Therefore, for $\gamma_A$ to be a constant we need $S_{at}^\lambda$ to grow at the same rate as $A_t^{\1-\phi}$.
	
	\item 
	
	Let's look at $\delta_t = ({B_t}^{\alpha} (1-l_t)(1-\sigma_t))^{-\beta}$. Note that if we assume that $l_t$ and $\sigma_t$ are constant, only $B_t$ changes with time.
	
	On the balanced growth path, it is straightforward to see that $B_t$ has positive growth and as such keeps increase. If $B_t\to \infty$ and $\alpha\beta>0$, $({B_t}^{\alpha} (1-l_t)(1-\sigma_t))^{-\beta} \to 0$, i.e. the mortality rate drops to 0 in the long-run.

	
	\item 
	\begin{enumerate}[(a)]
		\item 
		
		Note that if $\lambda_b>\lambda_a =\lambda$, we have that producing the good $H_t$ is faster. Hence, we get that the speed of convergence to a mortality rate of $0$ will also increase.
		
		Therefore, with $\lambda_b>\lambda_a =\lambda$, we get to the balanced growth path faster.
		
		\item
		
		With $\lambda_b>\lambda_a =\lambda$, we get to the steady state faster. As such the scientist will get to transfer to the consumption good side of research sooner (in the steady state), since there's no need for research of life saving good when the mortality rate is equal to 0. 
		
		Thus, we get improved production of consumption sooner and as such higher welfare. In the end, everyone is happy!
	\end{enumerate}
\end{enumerate}
	

\end{document}
