% Class Notes Template
\documentclass[12pt]{article}
\usepackage[margin=1in]{geometry} 
\usepackage[utf8]{inputenc}

% Packages
\usepackage[french, english]{babel}
\usepackage{amsmath,amsthm,amssymb,amsfonts, amsopn,mathrsfs, epstopdf, graphicx,color, tikz, float, enumerate}
\usepackage{listings}
\usepackage{color} %red, green, blue, yellow, cyan, magenta, black, white
\definecolor{mygreen}{RGB}{28,172,0} % color values Red, Green, Blue
\definecolor{mylilas}{RGB}{170,55,241}

\lstset{language=Matlab,%
	%basicstyle=\color{red},
	breaklines=true,%
	morekeywords={matlab2tikz},
	keywordstyle=\color{blue},%
	morekeywords=[2]{1}, keywordstyle=[2]{\color{black}},
	identifierstyle=\color{black},%
	stringstyle=\color{mylilas},
	commentstyle=\color{mygreen},%
	showstringspaces=false,%without this there will be a symbol in the places where there is a space
	numbers=left,%
	numberstyle={\tiny \color{black}},% size of the numbers
	numbersep=9pt, % this defines how far the numbers are from the text
	emph=[1]{for,end,break},emphstyle=[1]\color{blue}, %some words to emphasise
	%emph=[2]{word1,word2}, emphstyle=[2]{style},    
}

% Title
\title{ECON 6140 - Problem Set \# 2}
\date{\today}
\author{Julien Manuel Neves}

% Use these for theorems, lemmas, proofs, etc.
\newtheorem{theorem}{Theorem}
\newtheorem{corollary}[theorem]{Corollary}
\newtheorem{lemma}[theorem]{Lemma}
\newtheorem{observation}[theorem]{Observation}
\newtheorem{proposition}[theorem]{Proposition}
\newtheorem{definition}[theorem]{Definition}
\newtheorem{claim}[theorem]{Claim}
\newtheorem{fact}[theorem]{Fact}
\newtheorem{assumption}[theorem]{Assumption}
\newtheorem{problem}[theorem]{Problem}
\newtheorem{set-up}[theorem]{Set-up}
\newtheorem{example}[theorem]{Example}
\newtheorem{remark}[theorem]{Remark}
\newtheorem{axiom}[theorem]{Axiom}

% Usefuls Macros
\newcommand{\field}[1]{\mathbb{#1}}
\newcommand{\N}{\field{N}} % natural numbers
\newcommand{\R}{\field{R}} % real numbers
\newcommand{\Z}{\field{Z}} % integers
\newcommand\F{\mathcal{F}}
\newcommand\B{\mathbb{B}}
\renewcommand{\Re}{\R} % reals
\newcommand{\Rn}[1]{\mathbb{R}^{#1}}
\newcommand{\1}{{\bf 1}} % vector of all 1's
\newcommand{\I}[1]{\mathbb{I}_{\left\{#1\right\}}} % indicator function
\newcommand{\La}{\mathscr{L}}
% \newcommand{\tends}{{\rightarrow}} % arrow for limits
% \newcommand{\ra}{{\rightarrow}} % abbreviation for right arrow
% \newcommand{\subjectto}{\mbox{\rm subject to}} % subject to

%% math operators
\DeclareMathOperator*{\argmin}{arg\,min}
\DeclareMathOperator*{\argmax}{arg\,max}
\DeclareMathOperator*{\maximize}{maximize}
\DeclareMathOperator*{\minimize}{minimize}
\DeclareMathOperator{\E}{\mathsf{E}} % expectation
\newcommand{\Ex}[1]{\E\left\{#1\right\}} % expectation with brackets
\DeclareMathOperator{\pr}{\mathsf{P}} % probability
\newcommand{\prob}[1]{\pr\left\{#1\right\}}
\DeclareMathOperator{\subjectto}{{s.t.\ }} % subject to
\newcommand{\norm}[1]{\left\|#1\right\|}
\newcommand{\card}[1]{\left|#1\right|}

% Extra stuff
\newcommand\seq[1]{\{ #1 \}}
\newcommand{\inp}[2]{\langle #1, #2 \rangle}

\newcommand{\inv}{^{-1}}

\newcommand{\pa}[1]{\left(#1\right)}
\newcommand{\bra}[1]{\left[#1\right]}
\newcommand{\cbra}[1]{\left\{ #1 \right\}}

\newcommand{\pfrac}[2]{\pa{\frac{#1}{#2}}}
\newcommand{\bfrac}[2]{\bra{\frac{#1}{#2}}}

\newcommand{\mat}[1]{\begin{matrix}#1\end{matrix}}
\newcommand{\pmat}[1]{\pa{\mat{#1}}}
\newcommand{\bmat}[1]{\bra{\mat{#1}}}

\begin{document}

\maketitle

\section*{Problem 1}
\begin{enumerate}[(1)]
	\item
	
	Figure 1 shows the possible endowment paths.
	\begin{figure}[H]
		\centering
	\begin{tikzpicture}[level/.style={sibling distance=60mm/#1}]
	\node  [circle,draw]  (z){}
	child {node [circle,draw] (a) {$2$}
		child {node [circle,draw] (b) {$2$}
			child {node {$\vdots$} 
			} edge from parent[] node[left] {$\pi$}
		}
		child {node [circle,draw] (g) {$1$}
			child {node {$\vdots$}
			} edge from parent[] node[right] {$1-\pi$}
		} edge from parent[] node[left] {$\pi$}
	}
	child {node [circle,draw] (j) {$1$}
		child {node [circle,draw] (k) {$2$}
			child {node {$\vdots$} } edge from parent[] node[left] {$\pi$}
		} 
		child {node [circle,draw] (l) {$1$}
			child {node (c){$\vdots$}
				child {node (p) {} edge from parent[draw=none]
					child [grow=right] {node (q) {} edge from parent[draw=none]
						child [grow=right] {node (q) {} edge from parent[draw=none]
							child [grow=up] {node (r) {$\vdots$} edge from parent[draw=none]
								child [grow=up] {node (s) {$y_1$} edge from parent[draw=none]
									child [grow=up] {node (t) {$y_0$} edge from parent[draw=none]
										child [grow=up] {node (u) {} edge from parent[draw=none]}
									}
								}
							}
						}
					}
				}
			} edge from parent[] node[right] {$1-\pi$}
		} edge from parent[] node[right] {$1-\pi$}
	} ;
	\end{tikzpicture}
	
	\caption{Possible Endowment Paths}
		\end{figure}

	\item
	
	For $t\geq 1$, the Bellman equation is given by
	\begin{align*}
		V(a,y) & =\max_{c,a'}\ln(c) +\beta V(a',y')\\
		&\subjectto a' =(1+r)(a-c)+y'\\
		& \qquad a\geq c
	\end{align*}
	
	Note that for $t\geq 1$, $y'=y$ and therefore $c =\frac{y-a'}{1+r} +a$, thus
		\begin{align*}
	V(a,y)  =\max_{a'\geq y}\ln\left( \frac{y-a'}{1+r} +a\right)  +\beta V(a',y)
	\end{align*}
	
	Hence, the Euler equation for this problem is given by
	\[
	\frac{1}{c}\geq \beta (1+r)\frac{1}{c'}
	\]
	where it holds with equality if $a'\geq y$ ($a\geq c$) is not binding.
	
	Since $\beta(1+r)=1$, we get that Euler yields
	\[
	c'\geq c
	\]
	
	Let $c'=c=c^*$. If we can show that $a\geq c$ is not binding, then we are right to assume that $c'=c=c^*$. Looking at the budget constraint and imposing the no Ponzi scheme condition, we get that for $t\geq 1$,
	\begin{align*}
	a_t & = \sum_{j=0}^{\infty}\left( \frac{1}{1+r}\right)^j\left( c_{t+j} -\frac{y_{t+1+j}}{1+r} \right)  \\
	& = \sum_{j=0}^{\infty}\left( \frac{1}{1+r}\right)^j\left( c^* -\frac{y_{1}}{1+r} \right)  \\
	a_t & = \frac{1+r}{r} c^* -\frac{1}{r}y_1
	\end{align*}
	
	Hence, the assets holdings are constant. From the budget constraint, we have 
	\[
	a_t= a_1=(1+r)(a_0-c_0)+y_1=(1+r)(y_0-c_0)+y_1
	\]
	and therefore
		\begin{align*}
	a_1 & = \frac{1+r}{r} c^* -\frac{1}{r}y_1\\
	c^* & = \frac{1}{1+r}y_1 + \frac{r}{1+r}a_1\\
	& = \frac{1}{1+r}y_1 + \frac{r}{1+r}((1+r)(y_0-c_0)+y_1)\\
	\Rightarrow 	c^* & = y_1 + r(y_0-c_0)<a_t
	\end{align*}
	i.e.  $a\geq c$ is not binding.
	
	We can plug our results back into our Bellman equation to get the following value function
		\begin{align*}
V(a,y)  &=\ln\left( y_1 + r(y_0-c_0)\right)  +\beta V(a,y)\\
V(a,y) &= \frac{1}{1-\beta}\ln\left( y_1 + r(y_0-c_0)\right) \\
\Rightarrow V(a,y) &= \frac{1+r}{r}\ln\left( y_1 + r(y_0-c_0)\right)
\end{align*}

Thus, the problem at $t=0$ is the following
		\begin{align*}
w(y_0) &=\max_{c_0\geq y_0}\ln\left( c_0\right)  +\beta \Ex{\left( \frac{1+r}{r}\right) \ln\left( y_1 + r(y_0-c_0)\right)}\\
&=\max_{c_0\geq y_0}\ln\left( c_0\right)  + \frac{1}{r}\left[  \pi \ln\left( 2 + r(y_0-c_0)\right) +(1-\pi) \ln\left( 1 + r(y_0-c_0)\right)\right] \\
\end{align*}

Then, the FOCs yield
\[
\frac{1}{c_0}\geq \frac{\pi}{2 + r(y_0-c_0)} + \frac{1-\pi}{1 + r(y_0-c_0)}
\]
where it holds with equality if $y_0=c_0$

Now, let $y_0=2$. Can $y_0=c_0$? If $y_0=c_0=2$, then $\frac{1}{c_0}\geq \frac{\pi}{2 + r(y_0-c_0)} + \frac{1-\pi}{1 + r(y_0-c_0)}$ holds with inequality and
\[
\frac{1}{2}> \frac{\pi}{2} + \frac{1-\pi}{1} = \frac{2-\pi}{2} \geq \frac{1}{2}
\]
i.e. contradiction. Thus, we know that $a_0=y_0$ and we can solve for $c_0$ with
\[
\frac{1}{c_0}= \frac{\pi}{2 + r(y_0-c_0)} + \frac{1-\pi}{1 + r(y_0-c_0)}
\]

Finally, for $t\geq 1$, $a_t = (1+r)(a_0-c_0)+y_1=(1+r)(y_0-c_0)+y_1$ and $c_t = r(a_0-c_0)+y_1=r(y_0-c_0)+y_1$. 
	\item
	
	Let $y_0=1$. Can $y_0>c_0$? If $y_0>c_0$, then $\frac{1}{c_0}\geq \frac{\pi}{2 + r(y_0-c_0)} + \frac{1-\pi}{1 + r(y_0-c_0)}$ holds with equality and
	\[
	1<\frac{1}{c_0}= \frac{\pi}{2 + r(1-c_0)} + \frac{1-\pi}{1 + r(1-c_0)}< \frac{2-\pi}{2} \leq  1
	\]
	which is a contradiction.
	
	Hence, the constraint binds and $y_0=c_0=1$. In turn, we have for $t\geq 1$, $c_t=y_1$. 
	\item
	
	For $y_0=2$, we have
	\[
	u'(c^*) = \left\lbrace \mat{\frac{1}{r(2-c_0)+2} & \text{ with prob. }\pi\\
	\frac{1}{r(2-c_0)+1} & \text{ with prob. }(1-\pi)} \right. 
	\]
	
	For $y_0=1$, we have
	\[
	u'(c^*) = \left\lbrace \mat{\frac{1}{2} & \text{ with prob. }\pi\\
		1 & \text{ with prob. }(1-\pi)} \right. 
	\]
	\item
	
	In LS, we have the following proposition from Chamberlain and Wilson
	\begin{proposition}
		If there is an $\epsilon>0$ such that for any $\alpha\in \R_+$\[
		P(\alpha\leq y_t\leq \alpha +\epsilon\mid I_t)<1-\epsilon
		\]
		for all $I_t$ and $t\geq 0$, then $P(\lim_{t\to \infty}c_t =\infty) =1$.
	\end{proposition}

Let $\alpha = y_1$. Note that for any $t\geq 1$ and $\epsilon>0$, we have 
\[
P(\alpha\leq y_t\leq \alpha +\epsilon\mid I_t) = 1 \not< 1-\epsilon
\]

Hence, the conditions are not satisfied.
\end{enumerate}
\section*{Problem 2}
\begin{enumerate}[(1)]
	\item
	
	Let $y_t=y_1$ and $a_t=-\phi$. Then,
	\begin{align*}
		c_t+a_{t+1} & = (1+r)a_t+y_t\\
		c_t& = -(1+r)\frac{y_1}{r}+y_1 - a_{t+1}\\
		c_t& = -\phi - a_{t+1}\leq 0
	\end{align*}
	
	Hence, to prevent a situation where $a_{t+1}\geq -\phi$, we need $c_t>0$. Having the following Inada condition, $\lim_{c\to 0} u'(c)=\infty$, prevents the borrowing constraint when $y_t=y_1$.
	
	Now, let $y_{t-1}=y_i\in Y$. If there's a chance that in the next period we go from $i\to 1$, i.e. $y_t=y_1$, then the borrowing constraint can't hold with equality due to the previous argument. Thus, we require $\pi_{i1}>0$, i.e. a positive probability that from any $i$ we can go to $1$ in the next period.
	\item
	
	First, we carry over the previous conditions on $u$ and $\pi$ to this new problem.
	
	By iterating, the budget constraint and imposing a no Ponzi-scheme conditions, we get
		\begin{align*}
	a_t & = \sum_{j=0}^{\infty}\left( \frac{1}{1+r}\right)^j\left( c_{t+j} -y_{t+j} \right)  \\
	\Rightarrow a_t & = \sum_{j=0}^{\infty}\left( \frac{1}{1+r}\right)^j\left( \E_t\cbra{c_{t+j}} -\E_t\cbra{y_{t+j}} \right) 
	\end{align*}
	
	Similarly, we have
	\[
	a_{t+1}  = \sum_{j=0}^{\infty}\left( \frac{1}{1+r}\right)^j\left( \E_t\cbra{c_{t+1+j}} -\E_t\cbra{y_{t+1+j}} \right) 
	\]
	
	Hence, given $a_0>0$, if we can show $a_{t+1}\geq a_t \geq \dots \geq a_0>0$. 
	
	With no borrowing constraint, we have that the Euler equation holds with equality, i.e.
	\[
	u'(c_t) = \beta (1+r) \E_t\cbra{u'(c_{t+1})}
	\]
	
	First, we impose that $\beta(1+r)\geq 1$, hence
		\[
	u'(c_t) \geq  \E_t\cbra{u'(c_{t+1})}
	\]
	
	Next, we impose that $u$ is convex so that we can apply Jensen's inequality and get
			\[
	u'(c_t) \geq u'( \E_t\cbra{c_{t+1}})
	\]
	
	Finally, imposing that $u'$ is decreasing, we have 
				\[
	E_t\cbra{c_{t+1}} \geq c_t
	\]
	
	This in turns, implies that $c_t$ is sub-martingale and therefore
	\[
	\E_t\cbra{c_{t+1+j}} = \E_t\cbra{\E_{t+j}\cbra{c_{t+1+j}}} \geq \E_t\cbra{c_{t+j}}
	\]
	
	If have conditions on $\pi$ such that $y_t$ is a super martingale, we would get the following 
		\[
	\E_t\cbra{y_{t+1+j}} = \E_t\cbra{\E_{t+j}\cbra{y_{t+1+j}}} \leq \E_t\cbra{y_{t+j}}
	\]
	and
	\begin{align*}
	\E_t\cbra{c_{t+1+j}}-\E_t\cbra{y_{t+1+j}} &\geq \E_t\cbra{c_{t+j}}-\E_t\cbra{y_{t+j}}\\
		 \left( \frac{1}{1+r}\right)^j\left( \E_t\cbra{c_{t+1+j}}-\E_t\cbra{y_{t+1+j}}\right)  &\geq   \left( \frac{1}{1+r}\right)^j\left( \E_t\cbra{c_{t+j}}-\E_t\cbra{y_{t+j}} \right) \\
	 \sum_{j=0}^{\infty}\left( \frac{1}{1+r}\right)^j\left( \E_t\cbra{c_{t+1+j}}-\E_t\cbra{y_{t+1+j}}\right)  &\geq   \sum_{j=0}^{\infty}\left( \frac{1}{1+r}\right)^j\left( \E_t\cbra{c_{t+j}}-\E_t\cbra{y_{t+j}} \right) \\
	 a_{t+1} &\geq a_t
	\end{align*}
	i.e. the desired result.
	
\end{enumerate}
\section*{Problem 3}
\begin{enumerate}[(1)]
	\item
	
	First, note that
	\begin{align*}
	\ln{c_{t+1}} &\sim N(\mu_t,\nu_t)\\
	\Rightarrow -\gamma\ln{c_{t+1}} &\sim N(-\gamma\mu_t,\gamma^2\nu_t)
	\end{align*}
	
	This, implies that $\E_t \cbra{c_{t+1}^{-\gamma}}=e^{-\gamma u_t + \frac{1}{2}\gamma^2\nu_t}$.
	
	Then, the Euler equation turns out to be
	\begin{align*}
	u'(c_t) & =\beta R \E_t \cbra{u'(c_{t+1})}\\
	c_t^{-\gamma}& =\beta R \E_t \cbra{c_{t+1}^{-\gamma}}\\
	c_t^{-\gamma}& =\beta R \E_t \cbra{e^{-\gamma\ln{c_{t+1}}}}\\
	c_t^{-\gamma}& = \beta Re^{-\gamma \mu_t + \frac{1}{2}\gamma^2\nu_t}\\
	-\gamma\ln{c_t}& =\ln {\beta R }-\gamma \mu_t + \frac{1}{2}\gamma^2\nu_t\\
	-\gamma\ln{c_t}& =\ln {\beta R }-\gamma \E_t\cbra{\ln{c_{t+1}}} + \frac{1}{2}\gamma^2\nu_t\\
	\gamma\E_t\cbra{\ln{c_{t+1}}-\ln{c_t}}& = \ln {\beta R } + \frac{1}{2}\gamma^2\nu_t\\
	\Rightarrow \E_t\cbra{\Delta \ln{c_{t+1}}} &= \frac{1}{\gamma}\ln {\beta R } + \frac{1}{2}\gamma\nu_t
	\end{align*}
	
	\item
	
	Note that if we increase volatility, then $\E_t\cbra{\Delta \ln{c_{t+1}}} \uparrow$. This means that the expected consumption grows, implying that current saving must increase, i.e. precautionary saving behavior.
	
	\item
	First, we look at the regression
			\begin{align*}
	\Rightarrow \Ex{\E_t\cbra{\Delta \ln{c_{t+1}}}\mid y_t} &=  \Ex{\E_t\cbra{ \alpha_0+\alpha_1 y_t + \epsilon_{t+1}}\mid y_t} \\
	&=  \alpha_0+\alpha_1 y_t +\Ex{ \epsilon_{t+1}\mid y_t}
	\end{align*}
	Note that for the estimate of the previous regression to be consistent, we need $\Ex{\epsilon_{t+1}\mid y_t}=0$. 
	
	Looking at the previous model, where PIH holds, we have
		\begin{align*}
	\Rightarrow \Ex{\E_t\cbra{\Delta \ln{c_{t+1}}}\mid y_t} &=  \Ex{\frac{1}{\gamma}\ln {\beta R } + \frac{1}{2}\gamma\nu_t\mid y_t} \\
	\E_t\cbra{\Delta \ln{c_{t+1}}} &=\underbrace{ \frac{1}{\gamma}\ln {\beta R }}_{\alpha_0} +\underbrace{ 0}_{\alpha_1y_t} + \underbrace{\frac{1}{2}\gamma \Ex{\nu_t\mid y_t}}_{\Ex{\epsilon_{t+1}\mid y_t}}
	\end{align*}
	
Thus, if we have $\Ex{\nu_t\mid y_t}\neq 0$, the implied regression $\Delta \ln{c_{t+1}} = \alpha_0+\alpha_1 y_t + \epsilon_{t+1}$ has $\Ex{\epsilon_{t+1}\mid y_t}=0$. Hence, it would be possible to have $\alpha_1$ statistically different from 0, while the PIH still holds.
	
\end{enumerate}
\section*{Problem 4}

First, we start with the budget constraint,
\begin{align*}
	c_t+a_{t+1} &= Ra_t+y_t\\
	\Rightarrow a_{t+1} &= Ra_t+y_t-B(Ra_t+y_t)-D\\
	&= (1-B)(Ra_t+y_t)-D
\end{align*}

Ignoring borrowing limits and the fact that income is i.i.d., the Euler equation reduce to the following:
\begin{align*}
	u'(c_t) & =\beta R \E_t \cbra{u'(c_{t+1})}\\
	e^{-\sigma c_t} & =\beta R \E_t \cbra{e^{-\sigma c_{t+1}}}\\
	e^{-\sigma (B(Ra_t+y_t)+D)} & =\beta R \E_t \cbra{e^{-\sigma (B(Ra_{t+1}+y_{t+1})+D)}}\\
	e^{-\sigma (B(Ra_t+y_t)+D)} & =\beta R \E_t \cbra{e^{-\sigma (B(R ((1-B)(Ra_t+y_t)-D)+y_{t+1})+D) }}\\
	e^{-\sigma (B(Ra_t+y_t)+D)} & =\beta R \E_t \cbra{e^{-\sigma B y_{t+1}}} e^{-\sigma (B(R ((1-B)(Ra_t+y_t)-D))+D) }\\
	e^{-\sigma (B(Ra_t+y_t)+D)} & =\beta R \Ex{e^{-\sigma B y_{t+1}}} e^{-\sigma (BR (1-B)(Ra_t+y_t)+(1-RB)D) }\\
	e^{-\sigma B((1-R(1-B))(Ra_t+y_t)+RD)} & =\beta R \Ex{e^{-\sigma B y_{t+1}}}
\end{align*}

Hence, since the right side is a constant, we need 
\[
(1-R(1-B))=0 \Leftrightarrow B = 1-\frac{1}{R}
\]

Hence,
\begin{align*}
e^{-\sigma B((1-R(1-B))(Ra_t+y_t)+RD)} & =\beta R \Ex{e^{-\sigma B y_{t+1}}}\\
e^{-\sigma (R-1)D} & =\beta R \Ex{e^{-\sigma \left( \frac{R-1}{R}\right) y_{t+1}}}\\
-\sigma (R-1)D& = \ln{\beta R}+\ln{\Ex{e^{-\sigma \left( \frac{R-1}{R}\right) y_{t+1}}}}\\
\Rightarrow D& = -\frac{1}{\sigma (R-1)} \left[ \ln{\beta R}+\ln{\Ex{e^{-\sigma \left( \frac{R-1}{R}\right)  y_{t+1}}}}\right]
\end{align*}

Thus, since $R$ is constant, we have that $B$ and $D$ are constant and identical across agents. Recall that 
\[
a_{t+1} = (1-B)(Ra_t+y_t)-D
\]

How does $a_{t+1}$ change if we have mean preserving spread? Then $\Ex{e^{-\sigma \left( \frac{R-1}{R}\right)  y_{t+1}}}$ will increase and therefore $D\downarrow$ and finally $a_{t+1} \uparrow$. Since $D$ is the same for everyone, the change in $a_{t+1}$ will be identical across the distribution of $a_t$ and $y_t$.



\end{document}
